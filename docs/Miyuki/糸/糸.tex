\documentclass{article}
\usepackage{luatexja-fontspec}
\setmainjfont{Noto Serif JP}
\usepackage{luatexja-ruby}
\usepackage{multicol}
\usepackage{lipsum}
\usepackage{geometry}

\geometry{right=4.0cm,left=4.0cm}

\begin{document}

\setlength{\parindent}{0pt} % 取消段首缩进
\setlength{\parskip}{1em} % 设置段落间距


\section*{\ruby{糸}{いと}}

{\small 作詞・作曲 中島みゆき}\\

\begin{multicols}{2}

なぜめぐり\ruby{逢}{あ}うのかを\\
私たちはなにも\ruby{知}{し}らない\\
いつめぐり\ruby{逢}{あ}うのかを\\
私たちはいつも\ruby{知}{し}らない\\
\ruby{何処}{どこ}にいたの\ruby{生}{い}きてきたの\\
\ruby{遠}{とお}い\ruby{空}{そら}の\ruby{下}{した}\ruby{二}{ふた}つの\ruby{物語}{ものがたり}\\

\ruby{縦}{たて}の\ruby{糸}{いと}はあなた\\
\ruby{横}{よこ}の\ruby{糸}{いと}は私\\
\ruby{織}{お}りなす\ruby{布}{ぬの}はいつか誰かを\\
\ruby{暖}{あたた}めうるかも\ruby{知}{し}れない\\

なぜ\ruby{生}{い}きてゆくのかを\\
\ruby{迷}{まよ}った\ruby{日}{ひ}の\ruby{跡}{あと}のささくれ\\
\ruby{夢}{ゆめ}\ruby{追}{お}いかけ\ruby{走}{はし}って\\
ころんだ\ruby{日}{ひ}の\ruby{跡}{あと}のささくれ\\
こんな\ruby{糸}{いと}がなんになるの\\
\ruby{心}{こころ}\ruby{許}{もと}なくてふるえてた\ruby{風}{かぜ}の\ruby{中}{なか}\\

\ruby{縦}{たて}の\ruby{糸}{いと}はあなた\\
\ruby{横}{よこ}の\ruby{糸}{いと}は私\\
\ruby{織}{お}りなす\ruby{布}{ぬの}はいつか誰かを\\
\ruby{傷}{きず}をかばうかも\ruby{知}{し}れない\\

\ruby{縦}{たて}の\ruby{糸}{いと}はあなた\\
\ruby{横}{よこ}の\ruby{糸}{いと}は私\\
\ruby{逢}{あ}うべき\ruby{糸}{いと}に\ruby{出逢}{であ}えることを\\
\ruby{人}{ひと}は\ruby{仕合}{しあ}わせと\ruby{呼}{よ}びます\\


\end{multicols}





\end{document}
