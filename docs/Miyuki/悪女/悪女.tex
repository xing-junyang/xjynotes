\documentclass{article}
\usepackage{luatexja-fontspec}
\usepackage{luatexja-ruby}
\usepackage{multicol}
\usepackage{lipsum}
\usepackage{geometry}

\geometry{right=3.0cm,left=3.0cm}

\begin{document}

\setlength{\parindent}{0pt} % 取消段首缩进
\setlength{\parskip}{1em} % 设置段落间距

\ruby{}{}


{\Large\ruby{悪女}{あくじょ}}

{\small 作詞・作曲 中島みゆき}\\

\begin{multicols}{2}


\framebox{I}\\
\ruby{麻里子}{マリコ}の\ruby{部屋}{へや}へ\ruby{電話}{でんわ}をかけて\\
\ruby{男}{おとこ}と\ruby{遊}{あそ}んでる\ruby{芝居}{しばい} \ruby{続}{つず}けてきたけれど\\
あの\ruby{娘}{こ}も わりと\ruby{忙}{いそが}しいようで\\
そうそう つき\ruby{合}{あ}わせてもいられない\\

\ruby{土曜}{どよう}でなけりゃ\ruby{映画}{えいが}も\ruby{早}{はや}い\\
ホテルのロビーもいつまでいられるわけもない\\
\ruby{帰}{かえ}れるあてのあなたの\ruby{部屋}{へや}も\\
\ruby{受話器}{じゅわき}をはずしたままね \ruby{話}{はな}し\ruby{中}{ちゅう}\\

\framebox{II}\\
\ruby{女}{おんな}のつけぬ コロンを\ruby{買}{か}って\\
\ruby{深夜}{しんや}の サ\ruby{店}{てん}の\ruby{鏡}{かがみ}でうなじにつけたなら\\
\ruby{夜明}{よあ}けを\ruby{待}{ま}って \ruby{一}{いち}\ruby{番}{ばん}\ruby{電車}{でんしゃ}\\
\ruby{凍}{こご}えて\ruby{帰}{かえ}れば わざと\ruby{捨}{す}てゼリフ\\

\ruby{涙}{なみだ}も\ruby{捨}{す}てて \ruby{情}{なさけ}も捨てて\\
あなたが\ruby{早}{はや}く 私に\ruby{愛想}{あいそ}を\ruby{尽}{つ}かすまで\\
あなたの\ruby{隠}{かく}す あの\ruby{子}{こ}のもとへ\\
あなたを\ruby{早}{はや}く \ruby{渡}{わた}してしまうまで\\
\end{multicols}

\framebox{$\mathrm{C}_{\mathrm{horus}}$}\\
\ruby{悪女}{あくじょ}になるなら\ruby{月}{つき}\ruby{夜}{よ}はおよしよ\\
\ruby{素直}{すなお}になりすぎる\\
\ruby{隠}{かく}しておいた\ruby{言葉}{ことば}が ほろり\\
こぼれてしまう イカナイデ\\
\ruby{悪女}{あくじょ}になるなら\\
\ruby{裸足}{はだし}で\ruby{夜明}{よあ}けの\ruby{電車}{でんしゃ}で\ruby{泣}{な}いてから\\
\ruby{涙}{なみだ} ぽろぽろぽろぽろ\\
\ruby{流}{なが}れて \ruby{涸}{か}れでから\\

\framebox{順序 I $\mathrm{C}_{\mathrm{horus}}$ II $\mathrm{C}_{\mathrm{horus}}$}


\end{document}
