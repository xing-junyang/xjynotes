

\documentclass{article}
\usepackage{luatexja-fontspec}
\usepackage{fontspec}
\usepackage{luatexja-ruby}
\usepackage{multicol}
\usepackage{lipsum}
\usepackage{geometry}

\geometry{right=3.0cm,left=3.0cm}
\setmainjfont[
    BoldFont={Noto Serif CJK JP SemiBold},
]{Noto Serif CJK SC Regular} % 设置中文字体

\pagestyle{empty}

\begin{document}

\setlength{\parindent}{0pt} % 取消段首缩进
\setlength{\parskip}{1em} % 设置段落间距

\textbf{中文翻译\footnote{由本人自行整理与翻译,仅供辅助学习和欣赏原文用,如有纰漏望多多交流与斧正~}}

\section*{悪女}

{\small 作詞・作曲 中島みゆき}\\

\begin{multicols}{2}
\framebox{I}

往麻里子家里 拨了通电话\\
虽说还在继续着 同男人寻欢的戏码\\
但那姑娘看起来 似乎也挺忙的吧\\
罢了罢了 也不要指望她来陪我了\\

也不是周六 电影院早早就关了门\footnote{80年代日本电影院每逢周六通常放映到很晚,甚至通宵放映。}\\
酒店的大堂 也不能这样一直待到夜明\\
我可以去的地方 只剩下你那里了\\
电话听筒却耷拉着 空空回响着占线铃\\

\framebox{II}

买了女人不用的古龙香水\\
走进深夜茶店\footnote{即喫茶店(きっさてん),原歌词里简写为「サ店」,是一种日本特色的餐馆,出售咖啡、红茶等饮品以及和果子等简单食品。喫茶店和咖啡馆主要的区别在于经营许可证,一般而言喫茶店的营业许可更容易取得。} 对着镜子 在颈后抹开\\
然后等到天微亮时 搭第一班电车\\
拖着冻僵的心回去 再恶狠狠抛下几句伤人的对白\\

抛掉了泪水 也放下了情义\\
直到让你彻底厌倦 我曾给你的爱意\\
你一直深藏着的 另一个女孩\\
还是干脆早早把你 送去那里\\



\end{multicols}
\framebox{$\mathrm{C}_{\mathrm{horus}}$}

若要变成坏女人 切不要当月夜时分\\
被映得过分坦诚\\
句句压在心底的话 一扑簌\\
忍不住滴淌下来 「别离开」\\
若要变成坏女人\\
就光着脚 在黎明的电车上放声大哭\\
泪水扑扑落落\footnote{原歌词使用了一串拟声词「ぽろぽろぽろぽろ」,写眼泪大滴大滴滚落的声音。}\\
直到完全 流干\\

\framebox{演唱顺序 I $\triangleright$ $\mathrm{C}_{\mathrm{horus}}$ $\triangleright$ II $\triangleright$ $\mathrm{C}_{\mathrm{horus}}$}
\end{document}
