\documentclass{article}
\usepackage{luatexja-fontspec}
\usepackage{fontspec}
\usepackage{luatexja-ruby}
\usepackage{multicol}
\usepackage{lipsum}
\usepackage{geometry}

\geometry{right=3.0cm,left=3.0cm}
\setmainjfont[
    BoldFont={Noto Serif CJK JP SemiBold},
]{Noto Serif CJK JP Regular} % 设置中文字体

\pagestyle{empty}

\begin{document}

\setlength{\parindent}{0pt} % 取消段首缩进
\setlength{\parskip}{1em} % 设置段落间距

\ruby{}{}

\section*{\ruby{二}{に}\ruby{隻}{そう}の\ruby{舟}{ふね}\footnote{最初收錄於《East Asia》專輯中第八首,第二次收錄於《10 WINGS》專輯開頭。兩版編曲和歌詞次序略有不同,這裡採用《East Asia》版的編排。}}

{\small 作詞・作曲 中島みゆき}\\

\begin{multicols}{2}


\ruby{時}{とき}は \ruby{全}{すべ}てを\ruby{連}{つ}れてゆくものらしい\\
なのに どうして \ruby{寂}{さみ}しさを\ruby{置}{お}き\ruby{忘}{わす}れてゆくの\\
いくつになれば \ruby{人}{ひと}\ruby{懐}{なつ}かしさを\\
うまく\ruby{捨}{す}てられるようになるの\\
\ruby{難}{むずか}しいこと\ruby{望}{のぞ}んじゃいない\\
\ruby{有}{あ}り\ruby{得}{え}ないこと\ruby{望}{のぞ}んじゃいない\\
\ruby{時}{とき}よ \ruby{最後}{さいご}に\ruby{残}{のこ}してくれるなら\\
\ruby{寂}{さみ}しさの\ruby{分}{ぶん}だけ \ruby{愚}{おろ}かさをください\\

おまえとわたしは たとえば\ruby{二}{に}\ruby{隻}{そう}の\ruby{舟}{ふね}\\
\ruby{暗}{くら}い\ruby{海}{うみ}を\ruby{渡}{わた}ってゆく ひとつひとつの\ruby{舟}{ふね}\\
\ruby{互}{たが}いの\ruby{姿}{すがた}は\ruby{波}{なみ}に\ruby{隔}{へだ}てられても\\
\ruby{同}{おな}じ\ruby{歌}{うた}を歌いながらゆく \ruby{二}{に}\ruby{隻}{そう}の\ruby{舟}{ふね}\\

\ruby{時流}{じりゅう}を\ruby{泳}{およ}ぐ\ruby{海鳥}{うみどり}たちは\\
むごい\ruby{摂理}{せつり}をささやくばかり\\
いつかちぎれる\ruby{絆}{きずな}\ruby{見}{み}たさに\\
\ruby{高}{たか}く 高く 高く\\

\ruby{敢}{あ}えなくわたしが \ruby{波}{なみ}に\ruby{砕}{くだ}ける\ruby{日}{ひ}には\\
どこかでおまえの\ruby{舟}{ふね}が かすかにきしむだろう\\
それだけのことで わたしは\ruby{海}{うみ}をゆけるよ\\
たとえ\ruby{舫}{もや}い\ruby{網}{づな}は\ruby{切}{き}れて \ruby{嵐}{あらし}に\ruby{飲}{の}まれても\\\\
きこえてくるよ どんな\ruby{時}{とき}も\\

おまえの\ruby{悲鳴}{ひめい}が \ruby{胸}{むね}にきこえてくるよ\\
\ruby{越}{こ}えてゆけ と\ruby{叫}{さけ}ぶ\ruby{声}{こえ}が ゆくてを\ruby{照}{て}らすよ\\
おまえの\ruby{悲鳴}{ひめい}が \ruby{胸}{むね}にきこえてくるよ\\
\ruby{越}{こ}えてゆけ と\ruby{叫}{さけ}ぶ\ruby{声}{こえ}が ゆくてを\ruby{照}{て}らす\\\\

\ruby{難}{むずか}しいこと\ruby{望}{のぞ}んじゃいない\\
\ruby{有}{あ}り\ruby{得}{え}ないこと\ruby{望}{のぞ}んじゃいのに\\

\ruby{風}{かぜ}は\ruby{強}{つよ}く \ruby{波}{なみ}は\ruby{高}{たか}く\\
\ruby{闇}{やみ}は\ruby{深}{ふか}く \ruby{星}{ほし}も\ruby{見}{み}えない\\
\ruby{風}{かぜ}は\ruby{強}{つよ}く \ruby{波}{なみ}は\ruby{高}{たか}く\\
\ruby{暗}{くら}い\ruby{海}{うみ}は \ruby{果}{は}てるともなく\\
\ruby{風}{かぜ}の\ruby{中}{なか}で \ruby{波}{なみ}の\ruby{中}{なか}で\\
たかが\ruby{愛}{あい}は \ruby{木}{こ}の\ruby{葉}{は}のように\\
わたしたちは\ruby{二}{に}\ruby{隻}{そう}の\ruby{舟}{ふね}\\
ひとつずつの そしてひとつの\\
わたしたちは二隻の舟\\
ひとつずつの そしてひとつの\\
\textbf{わたしたちは 二隻の舟}\\


\end{multicols}

\end{document}
