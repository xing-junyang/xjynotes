\documentclass{article}
\usepackage{luatexja-fontspec}
\usepackage{fontspec}
\usepackage{luatexja-ruby}
\usepackage{multicol}
\usepackage{lipsum}
\usepackage{geometry}

\geometry{right=3.0cm,left=3.0cm}
\setmainjfont[
    BoldFont={Noto Serif CJK JP SemiBold},
]{Noto Serif CJK JP Regular} % 设置中文字体

\pagestyle{empty}

\begin{document}

\setlength{\parindent}{0pt} % 取消段首缩进
\setlength{\parskip}{1em} % 设置段落间距

\textbf{中文翻譯/中国語訳文}

\section*{兩艘船\footnote{原文為「二隻の舟」,常常寫為「二艘(そう)の舟」,意為兩艘船;這裡將「艘」替換為同音字「隻」,又包含了「成雙成對」的意思。}}

{\small 作詞・作曲 中島みゆき}\\

\begin{multicols}{2}


時光 據說可以帶走一切\\
可為何 偏偏遺下了寂寞呢\\
到底要經過 多少個春秋\\
對他人的依戀 才可以一點點釋懷呢\\
從未寄望於難園之夢\\
也不曾奢求那些無望之事\\
時光啊 如果最後能遺下些什麼的話\\
就請給我 同寂寞 不多不少的愚笨吧\\

你和我 就好比那海面上的兩艘小船\\
遠航於幽暗的汪洋之上 一艘又一艘的小船\\
即使彼此的身影 幾時被海浪隔斷\\
也是唱著同一首歌 緩緩遠去的兩艘船\\

在流轉的時代中沈浮的海鳥\\
只顧低聲訴說著殘酷的真理\\
為旁觀著我們 終將消散的牽絆\\
越飛越高 越高 越高\\

若有某日 悲慘的我 化為波濤裡的潔白浪花\\
海上某處 你漂流著的小船 也許會輕輕顫動吧\\
只需如此 我就能夠安心地駛向大海了\\
就算相系的船索寸斷 就算碎身於暴風雨之中\\
但無論何時 我都能聽到——\\

你的悲鳴 在我的心中清晰迴響\\
那衝破一切險阻的呼喊 照亮著我的前方\\
你的悲鳴 在我的心中清晰迴響\\
那衝破一切險阻的呼喊 照亮了我的前方啊\\

那些難園之夢 從來沒有去奢望過\\
那些無望之事 也從未奢望過啊 可是——\\

狂風大作 怒濤驚掀\\
無底的深暗中 不見一絲星點\\
暴風猛嘯 駭浪沖天\\
漆黑的海面上 一望無際無邊\\
在風的撕扯中 在浪的搖撼下\\
微不足道的愛啊 只如飄零的樹葉一般\\
你和我 是兩艘成對之船\\
縱使各自獨航 卻也永不分離\\
你和我 是兩艘成對之船\\
縱使各自獨航 卻也永不分離——\\
\textbf{我們是兩艘成對之船}\\


\end{multicols}

\end{document}
